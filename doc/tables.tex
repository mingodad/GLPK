%* tables.tex *%

%***********************************************************************
%  This code is part of GLPK (GNU Linear Programming Kit).
%
%  Copyright (C) 2000, 2001, 2002, 2003, 2004, 2005, 2006, 2007, 2008,
%  2009, 2010 Andrew Makhorin, Department for Applied Informatics,
%  Moscow Aviation Institute, Moscow, Russia. All rights reserved.
%  E-mail: <mao@gnu.org>.
%
%  GLPK is free software: you can redistribute it and/or modify it
%  under the terms of the GNU General Public License as published by
%  the Free Software Foundation, either version 3 of the License, or
%  (at your option) any later version.
%
%  GLPK is distributed in the hope that it will be useful, but WITHOUT
%  ANY WARRANTY; without even the implied warranty of MERCHANTABILITY
%  or FITNESS FOR A PARTICULAR PURPOSE. See the GNU General Public
%  License for more details.
%
%  You should have received a copy of the GNU General Public License
%  along with GLPK. If not, see <http://www.gnu.org/licenses/>.
%***********************************************************************

\documentclass[11pt,draft]{article}

\begin{document}

\title{Using Data Tables in the GNU MathProg Modeling Language}

\author{\begin{tabular}{cc}
Andrew Makhorin&Heinrich Schuchardt\\
{\tt <mao@gnu.org>}&{\tt<heinrich.schuchardt@gmx.de>}\\
\end{tabular}}

\date{November, 2009}

\maketitle

\begin{abstract}
This is a supplement to the document ``Modeling Language GNU
MathProg''. It describes the table statement, which allows reading data
from external tables into model objects such as sets and parameters as
well as writing data from the model to external tables.
\end{abstract}

\section*{Table statement}

\frame{
\begin{tabular}{@{\hspace*{6mm}}l@{\hspace*{17mm}}}
\\
{\tt table} {\it name} {\it alias} {\tt IN} {\it driver} {\it arg}
\dots {\it arg} {\tt:} \\
\hspace{10mm} {\it set}{\tt\ <- [} {\it fld} {\tt,} \dots {\tt,}
{\it fld} {\tt] ,}
{\it par} {\tt \symbol{126}} {\it fld} {\tt,} \dots {\tt,} {\it par}
{\tt \symbol{126}} {\it fld} {\tt;} \\
\\
{\tt table} {\it name} {\it alias} {\it domain} {\tt OUT} {\it driver}
{\it arg} \dots {\it arg} {\tt:} \\
\hspace{10mm} {\it expr} {\tt\symbol{126}} {\it fld} {\tt,}
\dots {\tt,} {\it expr} {\tt\symbol{126}} {\it fld} {\tt;}\\
\\
\end{tabular}
}

\medskip

\noindent
\begin{tabular}{@{}lp{110.2mm}@{}}
Where:&{\it name} is a symbolic name of the table;\\
&{\it alias} is an optional string literal which specifies the alias of
the table;\\
&{\it domain} is an indexing expression which specifies the subscript
domain of the (output) table;\\
&{\tt IN} is the keyword which means reading data from the input
table;\\
&{\tt OUT} is the keyword which means writing data to the output
table;\\
&{\it driver} is a symbolic expression which specifies the name of the
driver used to access the table. (For details see Section ``Table
drivers'' below.)\\
\end{tabular}

\noindent
\begin{tabular}{@{\hspace*{12.1mm}}lp{110.2mm}@{}}
&{\it arg} is an optional symbolic expression which is an argument
passed to the table driver. This symbolic expression must not contain
dummy indices specified in the domain;\\
&{\it set} is the name of an optional simple set called control set.
It can be omitted along with the delimiter `{\tt <-}';\\
&{\it fld} is the field name. Within square brackets at least one
field should be specified. The field name following parameter name or
expression is optional and can be omitted along with the delimiter
`{\tt\symbol{126}}', in which case the name of corresponding model
object is used as the field name;\\
&{\it par} is the symbolic name of a model parameter;\\
&{\it expr} is a numeric or symbolic expression.\\
\end{tabular}

\subsubsection*{Examples}

\begin{verbatim}
table data IN "CSV" "data.csv":
      s <- [FROM,TO], d~DISTANCE, c~COST;
table result{(f,t) in s} OUT "CSV" "result.csv":
      f~FROM, t~TO, x[f,t]~FLOW;
\end{verbatim}

The table statement allows reading data from a table into model objects
such as sets and (non-scalar) parameters as well as writing data from
the model to a table.

\subsection*{Table structure}

The {\it data table} is an (unordered) set of {\it records}, where each
record consists of the same number of {\it fields}, and each field is
provided with a unique symbolic name called the {\it field name}. For
example:

\bigskip

\begin{tabular}{@{\hspace*{43mm}}c@{\hspace*{11mm}}c@{\hspace*{10mm}}c
@{\hspace*{9mm}}c}
First&Second&&Last\\
field&field&.\ \ .\ \ .&field\\
$\downarrow$&$\downarrow$&&$\downarrow$\\
\end{tabular}

\begin{tabular}{ll@{}}
Table header&$\rightarrow$\\
First record&$\rightarrow$\\
Second record&$\rightarrow$\\
\\
\hfil .\ \ .\ \ .\\
\\
Last record&$\rightarrow$\\
\end{tabular}
\begin{tabular}{|l|l|c|c|}
\hline
{\tt FROM}&{\tt TO}&{\tt DISTANCE}&{\tt COST}\\
\hline
{\tt Seattle}  &{\tt New-York}&{\tt 2.5}&{\tt 0.12}\\
{\tt Seattle}  &{\tt Chicago} &{\tt 1.7}&{\tt 0.08}\\
{\tt Seattle}  &{\tt Topeka}  &{\tt 1.8}&{\tt 0.09}\\
{\tt San-Diego}&{\tt New-York}&{\tt 2.5}&{\tt 0.15}\\
{\tt San-Diego}&{\tt Chicago} &{\tt 1.8}&{\tt 0.10}\\
{\tt San-Diego}&{\tt Topeka}  &{\tt 1.4}&{\tt 0.07}\\
\hline
\end{tabular}

\subsection*{Reading data from input table}

The input table statement causes reading data from the specified table
record by record.

Once a next record has been read, numeric or symbolic values of fields,
whose names are enclosed in square brackets in the table statement, are
gathered into $n$-tuple, and if the control set is specified in the
table statement, this $n$-tuple is added to it. Besides, a numeric or
symbolic value of each field associated with a model parameter is
assigned to the parameter member identified by subscripts, which are
components of the $n$-tuple just read.

For example, the following input table statement:

\begin{verbatim}
table data IN "...": s <- [FROM,TO], d~DISTANCE, c~COST;
\end{verbatim}

\noindent
causes reading values of four fields named {\tt FROM}, {\tt TO},
{\tt DISTANCE}, and {\tt COST} from each record of the specified table.
Values of fields {\tt FROM} and {\tt TO} give a pair $(f,t)$, which is
added to the control set {\tt s}. The value of field {\tt DISTANCE} is
assigned to parameter member ${\tt d}[f,t]$, and the value of field
{\tt COST} is assigned to parameter member ${\tt c}[f,t]$.

Note that the input table may contain extra fields whose names are not
specified in the table statement, in which case values of these fields
on reading the table are ignored.

\subsection*{Writing data to output table}

The output table statement causes writing data to the specified table.
Note that some drivers (namely, CSV and xBASE) destroy the output table
before writing data, i.e. delete all its existing records.

Each $n$-tuple in the specified domain set generates one record written
to the output table. Values of fields are numeric or symbolic values of
corresponding expressions specified in the table statement. These
expressions are evaluated for each $n$-tuple in the domain set and,
thus, may include dummy indices introduced in the corresponding indexing
expression.

For example, the following output table statement:

\begin{verbatim}
table result{(f,t) in s} OUT "...": f~FROM, t~TO, x[f,t]~FLOW;
\end{verbatim}

\noindent
causes writing records, by one record for each pair $(f,t)$ in set
{\tt s}, to the output table, where each record consists of three
fields named {\tt FROM}, {\tt TO}, and {\tt FLOW}. The values written
to fields {\tt FROM} and {\tt TO} are current values of dummy indices
{\tt f} and {\tt t}, and the value written to field {\tt FLOW} is
a value of member ${\tt x}[f,t]$ of corresponding subscripted parameter
or variable.

\section*{Table drivers}

The {\it table driver} is a program module which provides transmitting
data between MathProg model objects and data tables.

Currently the GLPK package has four table drivers:

$\bullet$ built-in CSV table driver;

$\bullet$ built-in xBASE table driver;

$\bullet$ ODBC table driver;

$\bullet$ MySQL table driver.

\subsection*{CSV table driver}

The CSV table driver assumes that the data table is represented in the
form of a plain text file in the CSV (comma-separated values) file
format as described below.

To choose the CSV table driver its name in the table statement should
be specified as \verb|"CSV"|, and the only argument should specify the
name of a plain text file containing the table. For example:

\begin{verbatim}
table data IN "CSV" "data.csv": ... ;
\end{verbatim}

The filename suffix may be arbitrary, however, it is recommended to use
the suffix `\verb|.csv|'.

On reading input tables the CSV table driver provides an implicit field
named \verb|RECNO|, which contains the current record number. This field
can be specified in the input table statement as if there were the
actual field having the name \verb|RECNO| in the CSV file. For example:

\begin{verbatim}
table list IN "CSV" "list.csv": num <- [RECNO], ... ;
\end{verbatim}

\subsubsection*{CSV format\footnote{This material is based on the RFC
document 4180.}}

The CSV (comma-separated values) format is a plain text file format
defined as follows.

1. Each record is located on a separate line, delimited by a line
break. For example:

\begin{verbatim}
aaa,bbb,ccc\n
xxx,yyy,zzz\n
\end{verbatim}

\noindent
where `\verb|\n|' means the control character LF (0x0A).

\pagebreak

2. The last record in the file may or may not have an ending line
break. For example:

\begin{verbatim}
aaa,bbb,ccc\n
xxx,yyy,zzz
\end{verbatim}

3. There should be a header line appearing as the first line of the
file in the same format as normal record lines. This header should
contain names corresponding to the fields in the file. The number of
field names in the header line should be the same as the number of
fields in the records of the file. For example:

\begin{verbatim}
name1,name2,name3\n
aaa,bbb,ccc\n
xxx,yyy,zzz\n
\end{verbatim}

4. Within the header and each record there may be one or more fields
separated by commas. Each line should contain the same number of
fields throughout the file. Spaces are considered as part of a field
and therefore not ignored. The last field in the record should not
be followed by a comma. For example:

\begin{verbatim}
aaa,bbb,ccc\n
\end{verbatim}

5. Fields may or may not be enclosed in double quotes. For example:

\begin{verbatim}
"aaa","bbb","ccc"\n
zzz,yyy,xxx\n
\end{verbatim}

6. If a field is enclosed in double quotes, each double quote which is
part of the field should be coded twice. For example:

\begin{verbatim}
"aaa","b""bb","ccc"\n
\end{verbatim}

The following is a complete example of the data table in CSV format:

\begin{verbatim}
FROM,TO,DISTANCE,COST
Seattle,New-York,2.5,0.12
Seattle,Chicago,1.7,0.08
Seattle,Topeka,1.8,0.09
San-Diego,New-York,2.5,0.15
San-Diego,Chicago,1.8,0.10
San-Diego,Topeka,1.4,0.07
\end{verbatim}

\subsection*{xBASE table driver}

The xBASE table driver assumes that the data table is stored in the
.dbf file format.

To choose the xBASE table driver its name in the table statement should
be specified as \verb|"xBASE"|, and the first argument should specify
the name of a .dbf file containing the table. For the output table there
should be the second argument defining the table format in the form
\verb|"FF...F"|, where \verb|F| is either {\tt C({\it n})},
which specifies a character field of length $n$, or
{\tt N({\it n}{\rm [},{\it p}{\rm ]})}, which specifies a numeric field
of length $n$ and precision $p$ (by default $p$ is 0).

The following is a simple example which illustrates creating and
reading a .dbf file:

\begin{verbatim}
table tab1{i in 1..10} OUT "xBASE" "foo.dbf"
   "N(5)N(10,4)C(1)C(10)": 2*i+1 ~ B, Uniform(-20,+20) ~ A,
   "?" ~ FOO, "[" & i & "]" ~ C;
set S, dimen 4;
table tab2 IN "xBASE" "foo.dbf": S <- [B, C, RECNO, A];
display S;
end;
\end{verbatim}

\subsection*{ODBC table driver}

The ODBC table driver allows connecting to SQL databases using an
implementation of the ODBC interface based on the Call Level Interface
(CLI).\footnote{The corresponding software standard is defined in
ISO/IEC 9075-3:2003.}

\paragraph{Debian GNU/Linux.}
Under Debian GNU/Linux the ODBC table driver uses the iODBC
package,\footnote{See {\tt<http://www.iodbc.org/>}.} which should be
installed before building the GLPK package. The installation can be
effected with the following command:

\begin{verbatim}
sudo apt-get install libiodbc2-dev
\end{verbatim}

Note that on configuring the GLPK package to enable using the iODBC
library the option `\verb|--enable-odbc|' should be passed to the
configure script.

The individual databases must be entered for systemwide usage in
\linebreak \verb|/etc/odbc.ini| and \verb|/etc/odbcinst.ini|. Database
connections to be used by a single user are specified by files in the
home directory (\verb|.odbc.ini| and \verb|.odbcinst.ini|).

\paragraph{Microsoft Windows.}
Under Microsoft Windows the ODBC table driver uses the Microsoft ODBC
library. To enable this feature the symbol:

\begin{verbatim}
#define ODBC_DLNAME "odbc32.dll"
\end{verbatim}

\noindent
should be defined in the GLPK configuration file `\verb|config.h|'.

Data sources can be created via the Administrative Tools from the
Control Panel.

\bigskip

To choose the ODBC table driver its name in the table statement should
be specified as \verb|'ODBC'| or \verb|'iODBC'|.

The argument list is specified as follows.

The first argument is the connection string passed to the ODBC library,
for example:

\verb|'DSN=glpk;UID=user;PWD=password'|, or

\verb|'DRIVER=MySQL;DATABASE=glpkdb;UID=user;PWD=password'|.

Different parts of the string are separated by semicolons. Each part
consists of a pair {\it fieldname} and {\it value} separated by the
equal sign. Allowable fieldnames depend on the ODBC library. Typically
the following fieldnames are allowed:

\verb|DATABASE | database;

\verb|DRIVER   | ODBC driver;

\verb|DSN      | name of a data source;

\verb|FILEDSN  | name of a file data source;

\verb|PWD      | user password;

\verb|SERVER   | database;

\verb|UID      | user name.

The second argument and all following are considered to be SQL
statements

SQL statements may be spread over multiple arguments.  If the last
character of an argument is a semicolon this indicates the end of
a SQL statement.

The arguments of a SQL statement are concatenated separated by space.
The eventual trailing semicolon will be removed.

All but the last SQL statement will be executed directly.

For IN-table the last SQL statement can be a SELECT command starting
with the capitalized letters \verb|'SELECT '|. If the string does not
start with \verb|'SELECT '| it is considered to be a table name and a
SELECT statement is automatically generated.

For OUT-table the last SQL statement can contain one or multiple
question marks. If it contains a question mark it is considered a
template for the write routine. Otherwise the string is considered a
table name and an INSERT template is automatically generated.

The writing routine uses the template with the question marks and
replaces the first question mark by the first output parameter, the
second question mark by the second output parameter and so forth. Then
the SQL command is issued.

The following is an example of the output table statement:

\begin{small}
\begin{verbatim}
table ta { l in LOCATIONS } OUT
   'ODBC'
   'DSN=glpkdb;UID=glpkuser;PWD=glpkpassword'
   'DROP TABLE IF EXISTS result;'
   'CREATE TABLE result ( ID INT, LOC VARCHAR(255), QUAN DOUBLE );'
   'INSERT INTO result 'VALUES ( 4, ?, ? )' :
   l ~ LOC, quantity[l] ~ QUAN;
\end{verbatim}
\end{small}

\noindent
Alternatively it could be written as follows:

\begin{small}
\begin{verbatim}
table ta { l in LOCATIONS } OUT
   'ODBC'
   'DSN=glpkdb;UID=glpkuser;PWD=glpkpassword'
   'DROP TABLE IF EXISTS result;'
   'CREATE TABLE result ( ID INT, LOC VARCHAR(255), QUAN DOUBLE );'
   'result' :
   l ~ LOC, quantity[l] ~ QUAN, 4 ~ ID;
\end{verbatim}
\end{small}

Using templates with `\verb|?|' supports not only INSERT, but also
UPDATE, DELETE, etc. For example:

\begin{small}
\begin{verbatim}
table ta { l in LOCATIONS } OUT
   'ODBC'
   'DSN=glpkdb;UID=glpkuser;PWD=glpkpassword'
   'UPDATE result SET DATE = ' & date & ' WHERE ID = 4;'
   'UPDATE result SET QUAN = ? WHERE LOC = ? AND ID = 4' :
   quantity[l], l;
\end{verbatim}
\end{small}

\subsection*{MySQL table driver}

The MySQL table driver allows connecting to MySQL databases.

\paragraph{Debian GNU/Linux.}
Under Debian GNU/Linux the MySQL table\linebreak driver uses the MySQL
package,\footnote{For download development files see
{\tt<http://dev.mysql.com/downloads/mysql/>}.} which should be installed
before building the GLPK package. The installation can be effected with
the following command:

\begin{verbatim}
sudo apt-get install libmysqlclient15-dev
\end{verbatim}

Note that on configuring the GLPK package to enable using the MySQL
library the option `\verb|--enable-mysql|' should be passed to the
configure script.

\paragraph{Microsoft Windows.}
Under Microsoft Windows the MySQL table driver also uses the MySQL
library. To enable this feature the symbol:

\begin{verbatim}
#define MYSQL_DLNAME "libmysql.dll"
\end{verbatim}

\noindent
should be defined in the GLPK configuration file `\verb|config.h|'.

\bigskip

To choose the MySQL table driver its name in the table statement should
be specified as \verb|'MySQL'|.

The argument list is specified as follows.

The first argument specifies how to connect the data base in the DSN
style, for example:

\verb|'Database=glpk;UID=glpk;PWD=gnu'|.

Different parts of the string are separated by semicolons. Each part
consists of a pair {\it fieldname} and {\it value} separated by the
equal sign. The following fieldnames are allowed:

\verb|Server   | server running the database (defaulting to localhost);

\verb|Database | name of the database;

\verb|UID      | user name;

\verb|PWD      | user password;

\verb|Port     | port used by the server (defaulting to 3306).

The second argument and all following are considered to be SQL
statements

SQL statements may be spread over multiple arguments.  If the last
character of an argument is a semicolon this indicates the end of
a SQL statement.

The arguments of a SQL statement are concatenated separated by space.
The eventual trailing semicolon will be removed.

All but the last SQL statement will be executed directly.

For IN-table the last SQL statement can be a SELECT command starting
with the capitalized letters \verb|'SELECT '|. If the string does not
start with \verb|'SELECT '| it is considered to be a table name and a
SELECT statement is automatically generated.

For OUT-table the last SQL statement can contain one or multiple
question marks. If it contains a question mark it is considered a
template for the write routine. Otherwise the string is considered a
table name and an INSERT template is automatically generated.

The writing routine uses the template with the question marks and
replaces the first question mark by the first output parameter, the
second question mark by the second output parameter and so forth. Then
the SQL command is issued.

The following is an example of the output table statement:

\begin{small}
\begin{verbatim}
table ta { l in LOCATIONS } OUT
   'MySQL'
   'Database=glpkdb;UID=glpkuser;PWD=glpkpassword'
   'DROP TABLE IF EXISTS result;'
   'CREATE TABLE result ( ID INT, LOC VARCHAR(255), QUAN DOUBLE );'
   'INSERT INTO result VALUES ( 4, ?, ? )' :
   l ~ LOC, quantity[l] ~ QUAN;
\end{verbatim}
\end{small}

\noindent
Alternatively it could be written as follows:

\begin{small}
\begin{verbatim}
table ta { l in LOCATIONS } OUT
   'MySQL'
   'Database=glpkdb;UID=glpkuser;PWD=glpkpassword'
   'DROP TABLE IF EXISTS result;'
   'CREATE TABLE result ( ID INT, LOC VARCHAR(255), QUAN DOUBLE );'
   'result' :
   l ~ LOC, quantity[l] ~ QUAN, 4 ~ ID;
\end{verbatim}
\end{small}

Using templates with `\verb|?|' supports not only INSERT, but also
UPDATE, DELETE, etc. For example:

\begin{small}
\begin{verbatim}
table ta { l in LOCATIONS } OUT
   'MySQL'
   'Database=glpkdb;UID=glpkuser;PWD=glpkpassword'
   'UPDATE result SET DATE = ' & date & ' WHERE ID = 4;'
   'UPDATE result SET QUAN = ? WHERE LOC = ? AND ID = 4' :
   quantity[l], l;
\end{verbatim}
\end{small}

%\vspace{15mm}
%
%\begin{center}
%\rule{40mm}{.5pt}
%\end{center}

\end{document}
