%* glpk07.tex *%

\chapter{Miscellaneous API Routines}

\section{Library environment routines}

\subsection{glp\_long---64-bit integer data type}

Some GLPK API routines use 64-bit integer data type, which is declared
in the header \verb|glpk.h| as follows:

\begin{verbatim}
typedef struct { int lo, hi; } glp_long;
\end{verbatim}

\noindent
where \verb|lo| contains low 32 bits, and \verb|hi| contains high 32
bits of 64-bit integer value.\footnote{GLPK conforms to ILP32, LLP64,
and LP64 programming models, where the built-in type {\tt int}
corresponds to 32-bit integers.}

\subsection{glp\_version---determine library version}

\subsubsection*{Synopsis}

\begin{verbatim}
const char *glp_version(void);
\end{verbatim}

\subsubsection*{Returns}

The routine \verb|glp_version| returns a pointer to a null-terminated
character string, which specifies the version of the GLPK library in
the form \verb|"X.Y"|, where `\verb|X|' is the major version number, and
`\verb|Y|' is the minor version number, for example, \verb|"4.16"|.

\subsubsection*{Example}

\begin{verbatim}
printf("GLPK version is %s\n", glp_version());
\end{verbatim}

\subsection{glp\_term\_out---enable/disable terminal output}

\subsubsection*{Synopsis}

\begin{verbatim}
void glp_term_out(int flag);
\end{verbatim}

\subsubsection*{Description}

Depending on the parameter flag the routine \verb|glp_term_out| enables
or disables terminal output performed by glpk routines:

\verb|GLP_ON |---enable terminal output;

\verb|GLP_OFF|---disable terminal output.

\subsection{glp\_term\_hook---intercept terminal output}

\subsubsection*{Synopsis}

\begin{verbatim}
void glp_term_hook(int (*func)(void *info, const char *s),
      void *info);
\end{verbatim}

\subsubsection*{Description}

The routine \verb|glp_term_hook| installs the user-defined hook routine
to intercept all terminal output performed by GLPK routines.

%This feature can be used to redirect the terminal output to other
%destination, for example, to a file or a text window.

The parameter {\it func} specifies the user-defined hook routine. It is
called from an internal printing routine, which passes to it two
parameters: {\it info} and {\it s}. The parameter {\it info} is a
transit pointer specified in corresponding call to the routine
\verb|glp_term_hook|; it may be used to pass some additional information
to the hook routine. The parameter {\it s} is a pointer to the null
terminated character string, which is intended to be written to the
terminal. If the hook routine returns zero, the printing routine writes
the string {\it s} to the terminal in a usual way; otherwise, if the
hook routine returns non-zero, no terminal output is performed.

To uninstall the hook routine both parameters {\it func} and {\it info}
should be specified as \verb|NULL|.

\subsubsection*{Example}

\begin{verbatim}
static int hook(void *info, const char *s)
{     FILE *foo = info;
      fputs(s, foo);
      return 1;
}

int main(void)
{     FILE *foo;
      . . .
      /* redirect terminal output */
      glp_term_hook(hook, foo);
      . . .
      /* resume terminal output */
      glp_term_hook(NULL, NULL);
      . . .
}
\end{verbatim}

\subsection{glp\_mem\_usage---get memory usage information}

\subsubsection*{Synopsis}

\begin{verbatim}
void glp_mem_usage(int *count, int *cpeak, glp_long *total,
      glp_long *tpeak);
\end{verbatim}

\subsubsection*{Description}

The routine \verb|glp_mem_usage| reports some information about
utilization of the memory by GLPK routines. Information is stored to
locations specified by corresponding parameters (see below). Any
parameter can be specified as \verb|NULL|, in which case corresponding
information is not stored.

\verb|*count| is the number of currently allocated memory blocks.

\verb|*cpeak| is the peak value of \verb|*count| reached since the
initialization of the GLPK library environment.

\verb|*total| is the total amount, in bytes, of currently allocated
memory blocks.

\verb|*tpeak| is the peak value of \verb|*total| reached since the
initialization of the GLPK library envirionment.

\subsubsection*{Example}

\begin{verbatim}
glp_mem_usage(&count, NULL, NULL, NULL);
printf("%d memory block(s) are still allocated\n", count);
\end{verbatim}

\subsection{glp\_mem\_limit---set memory usage limit}

\subsubsection*{Synopsis}

\begin{verbatim}
void glp_mem_limit(int limit);
\end{verbatim}

\subsubsection*{Description}

The routine \verb|glp_mem_limit| limits the amount of memory available
for dynamic allocation (in GLPK routines) to \verb|limit| megabytes.

\subsection{glp\_free\_env---free GLPK library environment}

\subsubsection*{Synopsis}

\begin{verbatim}
void glp_free_env(void);
\end{verbatim}

\subsubsection*{Description}

The routine \verb|glp_free_env| frees all resources used by GLPK
routines (memory blocks, etc.) which are currently still in use.

\subsubsection*{Usage notes}

Normally the application program does not need to call this routine,
because GLPK routines always free all unused resources. However, if
the application program even has deleted all problem objects, there
will be several memory blocks still allocated for the internal library
needs. For some reasons the application program may want GLPK to free
this memory, in which case it should call \verb|glp_free_env|.

Note that a call to \verb|glp_free_env| invalidates all problem objects
which still exist.

%* eof *%
